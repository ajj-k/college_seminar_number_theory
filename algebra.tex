\documentclass{jsarticle}

\usepackage{amsmath}
\usepackage{amssymb}
% 定義についてのパッケージ
\usepackage{amsthm}
\newtheorem{dfn}{Def}
\newtheorem{thm}{Thm}
\newtheorem*{prop}{Prop}
\newtheorem{cor}{Cor}
\newtheorem*{lem}{Lem}

\title{ゼミレポート}
\author{kiyoshi ohashi}
\date{2023年4月7日}

\begin{document}

    \maketitle

    次に,正の整数が素数であるかを判定する最古のアルゴリズムである\textbf{エラトステネスのふるい}について確認しよう.

    % 命題のあとのピリオドをなくしたい
    \begin{prop}{\rm\bf 1.4.9}
      $n>1$が合成数ならば,$\sqrt{n}$以下の素数の約数を持つ
    \end{prop}

    \begin{proof}
      $n$が合成数と仮定すると,$^{\exists}l \in \mathbb{Z}$ s.t. $l \mid n$ である.
      ただし,$n>1$の合成数より,$ l \neq 1, n$.ここで,$m=n/l$とおくと,$l,m \neq 1,n$,$n = lm$である.(合成数の定義から$l>0$).
      もし,$l,m > \sqrt{n}$であるとき,$lm > n$となり矛盾する.
      よって,$l \leq \sqrt{n}$ または $m \leq \sqrt{n}$ となるため,$n$は$\sqrt{n}$以下の約数$l$を持つ. \\
      \textbf{Prop 1.4.8} より,$l$は素数の約数を持ち,それを$p$と置くと$ p \mid l$であり,\textbf{Prop 1.4.7} より,推移律から$p \mid n$となる. \\
      したがって,$p \leq l \leq \sqrt{n}$であるから,$n$は$\sqrt{n}$以下の素数の約数を持つ.
    \end{proof}

    この命題の対偶を取れば,次のような表現となる

    \begin{prop}{\rm\bf 1.4.9${}^\text{$\prime$}$}
      $\sqrt{n}$以下の素数の約数を持たないならば,$n$は素数である.
    \end{prop}

    「割り切れる(割り算)」の概念を規定するのは\textbf{Prop 1.4.15}とまだ先であるが,その概念を用いて説明するならば,
    本命題が主張することは ある正の整数$n$が与えられた時,$\sqrt{n}$以下の素数全てで$n$を割り切れない時,$n$が素数であることを主張する.
    すなわち,$n$が素数であることを調べる際に,$n$回ではなく$\sqrt{n}$回のステップのみで充分であることを意味する. \\

    また,次の補題についても確認しよう.

    \begin{lem}{\rm\bf 1.4.10}
      $m, n \in \mathbb{Z}$のとき,$n \mid m$, $n \neq \pm 1$ $\implies$ $n \nmid m + 1$
    \end{lem}

    \begin{proof}
      仮定より$^{\exists}a \in \mathbb{Z}$ s.t. $m = na$である.
      もし,$n \mid m+1$ならば,$^{\exists}b \in \mathbb{Z}$ s.t. $m+1 = nb$であり,
      $1 = (m+1)-m = n(b-a)$である.\textbf{Cor 1.4.3} より,$n = \pm 1$であるが,これは矛盾
    \end{proof}

    今回のゼミにおいて,個人的に疑問に思ったのは $n \mid m$ によって $n \neq \pm 1$ $\implies$ $n \nmid m + 1$ を束縛しているのではないか,という点である.
    この事については,束縛させても仮定に用いても,帰結されるものに変わりはない.
    むしろ,今回の場合には$n \mid m+1$を仮定した際に,$n \nmid m$ または $n = \pm 1$を考え,それぞれの命題変数の成立の可否をを調べることが,考察の手立てとなる.

\end{document}