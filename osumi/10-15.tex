


% \documentclass{jsarticle}
% \usepackage{amsmath}
% \usepackage{amssymb}
% \usepackage{amsfonts}
% \usepackage{otf}
% \usepackage{import}
% \usepackage{cleveref}

% \usepackage{mathtools}
% \DeclarePairedDelimiter{\abs}{\lvert}{\rvert} % | | 絶対値
% \DeclarePairedDelimiter{\norm}{\lVert}{\rVert} % || || norm
% \DeclarePairedDelimiter{\rbra}{\lparen}{\rparen} % () 丸括弧
% \DeclarePairedDelimiter{\cbra}{\lbrace}{\rbrace} % {} 中括弧
% \DeclarePairedDelimiter{\sbra}{\lbrack}{\rbrack} % [] 角括弧
% \DeclarePairedDelimiter{\abra}{\langle}{\rangle} % < > 山括弧
% \DeclarePairedDelimiter{\floor}{\lfloor}{\rfloor} % floor function
% \DeclarePairedDelimiter{\ceil}{\lceil}{\rceil}

% \DeclareMathOperator{\lcm}{lcm}% lcm 最小公倍数
% \usepackage{amsthm}


% \theoremstyle{definition}
% \newtheorem{dfn}{Definition}
% \newtheorem{thm}{Theorem}
% \newtheorem{prop}{Proposition}
% \newtheorem{cor}{Corolary}
% \newtheorem{lem}{Lemma}
% \newtheorem{axiom}{Axiom}
% \newcommand{\integer}{\mathbb{Z}}
% \newcommand{\nat}{\mathbb{N}}
% \crefname{thm}{Theorem}{Theorems}








\section{整数の基本性質}

$\mathbb{Z}$には和と積が定義でき, 結合法則・交換法則・分配法則が成り立つ. つまり$a, b, c\in\mathbb{Z}$ならば
\vskip\baselineskip
\begin{center}
$(a+b)+c=a+(b+c), (ab)c=a(bc), $
\\$a+b=b+a, ab=ba, $
\\ $a+0=0+a=a, a1=1a=a, $
\\ $a(b+c)=ab+ac$
\end{center}
\vskip\baselineskip
である. 
\vskip\baselineskip
$x\in\mathbb{Z}$は$x\in\mathbb{N}$であるときに$x\geqq0$と定義する. 
さらに$x\geqq0$であって$x\ne0$であるものを正の整数, 同様に$x\leq0$であって$x\ne0$であるものを負の整数と定義する. 

$x, y\in\mathbb{Z}, x-y>0(x-y\geqq0)$なら, $x>y(x\geqq y)$と定義する. 
\\$x, y\in\mathbb{Z}$なら, $x>y, x=y, x<y$のどれか一つが必ず成り立つ. $x\geqq y$なら, 
$\max{\cbra{x, y}} = x, \min{\cbra{x, y}} = y$と定義する. 
\\$x\leq y$である場合も同様で, $n$が正の整数ならば, $n$以下の正の整数の数は有限であることを認める. 
\\整数の大小関係について以下が成り立つ. 


$a, b, c\in\mathbb{Z}, c>0$ならば
\begin{equation*}\label{1.3.1}
a>b\iff a+c>b+c \iff ac\ne bc
\end{equation*}
が成り立つ. 
\\なお, $c<0$なら$ac<bc$である. 特に$c\ne0$ならば, 
\begin{equation*} \label{1.3.2}
a\ne b \iff ac\ne bc
\end{equation*}

\vskip\baselineskip
また$x\in\mathbb{R}$に対して, 
\begin{equation}
|x|=
\begin{cases}x & \text{$x\geqq0$}
\\-x & \text{$x \leq 0$}
\nonumber
\end{cases}
\end{equation}
と絶対値を定義する. $n$が正の整数なら, $|x|\leq n$である整数の個数は$2n+1$個であり, 有限である. 
\vskip\baselineskip
公理として次の性質を述べる. 

\begin{axm}\label{1.3.3}
1は最小の正の整数である.
\end{axm}
この公理を認めると, $n\in\mathbb{N}, n<m$である最小の
$m$が$n-m>0$より, $n-m=1$を満たす. よって$n$より大きい整数の中で最小のものは$n+1$とわかる。

\begin{prop}\label{1.3.4}
$a>0$を整数とするとき, 次の(1)~(3)が成り立つ
\begin{enumerate}
\renewcommand{\labelenumi}{(\arabic{enumi})}
\item$b\in\mathbb{Z}$, $|b-a|<1$なら, $b=a$である. 
\item$|b|<1$なら, $b=0$である. 
\item$b, b'\in\mathbb{Z}, 0\leq b, b'\leq a, $なら, $|b-b'|<a$

\end{enumerate}
\end{prop}

\begin{proof}
(1)\quad $c=b-a$とおく, $c\in\mathbb{Z}, |c|<1$である, $c>0$なら$0<c<1$となり, この整数$c$は正の1より小さい整数なので, \cref{1.3.3}に矛盾する. $c<0$ならば$-1<c<0$なので$0<-c<1$となりこれも\cref{1.3.3}に矛盾する. 従って$c=0, $つまり$b=a$である. 
\\(2)\quad $|b|<1$で$b>0$なら$0<b<1$となり. \cref{1.3.3}に矛盾, と(1)と同様に$b=0$が示される. また, この命題「$|b|<1$なら$b=0$」の対偶をとると「$b\ne0$ならば$|b|\geqq1$」も導かれる. 
\\(3)\quad $b'\geqq0$と$b<a$より$b-b' \leq b<a$, また, $b\geqq0$と$b'<a$より$b-b'\geqq-b'>-a$従って$-a<b-b'<a$となる. 
\end{proof}

\begin{axm}\label{1.3.5}
アルキメデスの公理 \quad $x>0$が実数なら, $x$以下の正の整数の数は有限個である. 
\end{axm}

このアルキメデスの公理と同値な表現として、次のような表現も耳にしたことがあるだろう
\begin{axm}\label{1.3.5.1}
\({}^\forall a, b , {}^\exists n, \mathrm{s.t.} a<nb\)
\end{axm}
この公理をアルキメデスの公理とし, \cref{1.3.5}との同値性を確認する. そのためにこの補題を用意する. 

\begin{lem}\label{1.3.5.2}
$x>0$が実数なら, $x$以下の正の整数の個数は有限個である.
\end{lem}


\begin{prop}
$x>0$が実数なら$x$より真に大きい正の整数が存在する.
\end{prop}

\begin{proof}
\begin{enumerate}
\renewcommand{\labelenumi}{(\arabic{enumi})}

\item \cref{1.3.5.1} ならば \cref{1.3.5.2}の証明
\\ \cref{1.3.5.1}より,$x$以下の正の整数の中に最大のものが存在する. それを$x_1$とおくと, $x_1+1\in \integer$である. もし, $x\geqq x_1+1$とすると, $x_1+1\leq x_1$となって, $x_1$の定義である最大性に反するので矛盾. よって$x<x_1+1$から\cref{1.3.5.2}が導かれる.
\item \cref{1.3.5.2}ならば\cref{1.3.5}の証明
\\ \cref{1.3.5.2}より実数$\dfrac{a}{b}$より大きい正整数$n$が存在する. つまり$\dfrac{a}{b}<n$, 従って$a<nb$.
\item \cref{1.3.5}ならば\cref{1.3.5.2}の証明
\\ \cref{1.3.5}より\({}^\forall a, b , {}^\exists n, \mathrm{s.t.} \dfrac{a}{b}<n\)が言える. ここで$b=1$とすると, \({}^\forall a\in\mathbb{R_>},  {}^\exists n, \mathrm{s.t.} a<n\).よって\cref{1.3.5.2}が導かれる. 
\item \cref{1.3.5.2}ならば\cref{1.3.5.1}
\\$x>0$を実数とする. \cref{1.3.5.2}より$x<n$を満たす正整数$n$が存在する. $\{x_1\in\integer_>|x_1<x\}\subset\{x_2\in\integer_>|x_2<x\}$
\end{enumerate}
(1)~(4)により\cref{1.3.5.1}と\cref{1.3.5}の同値性が証明された.
\end{proof}

\cref{1.3.5}を認めると, 正の実数$x$に対して, $x$以下の整数で最大のものが存在すると分かる. これを[$x$]と書く. [$x$]の定義より, $x<[x]+1$である. $y\in\mathbb{Z}$で$x<y$ならば, $[x]<y$なので$[x]+1\leq y$である. 
逆に$[x]+1\leq y$なら, $x<[x]+1$なので, , $x<[x]+1\leq yからx<y$である. よって$y=[x]+1$が$x<y$である最小の整数である. 負の場合も$x$以下の最大の整数が存在することがわかる. この$[x]$をガウス記号と呼ぶ. 
$[x]$の定義より, $[x]\leq x<[x]+1$である. $x>0$の場合と同様に$y=[x]+1$は$y>x$である最小の整数である. 


\section{整数の合同}

\begin{dfn}\label{1.4.1}
$a, b\in\mathbb{Z}$とする. 
\\(1)$a\ne0$であるとき, $b=an$となる$n\in\mathbb{Z}$があるなら, $b$を$a$の倍数, $a$を$b$の約数といい, $a|b$と書く. $b$が$a$の倍数でないなら, $a\not|b$と書く
\\(2)$a, b$の共通の約数を$a, b$ の公約数と呼ぶ
\\(3)$a\ne0$かつ$b\ne0$であるとき, $a, b$の共通の倍数を$a, b$の公倍数という
\end{dfn}

\begin{prop}\label{1.4.2}
$a, b\in\mathbb{Z}\setminus {{0}}$で$b|a$ならば, $|b|\leq|a|$である. よって$a$の約数の個数は有限である
$a=bn$となる$n\in\mathbb{N}$がある. $a\ne0$なので, $n\ne0$である. $|n|>0$は整数なので, \cref{1.3.3}より$|n|\geqq1$である, よって, $|a|=|b||n|\geqq|b|$である
\end{prop}


絶対値が$1$以下である整数は, $\pm1, 0$だけなので, 次の系を得る. 

\begin{cor}\label{1.4.3}
\quad $1$の約数は$\pm1$である
\end{cor}

\begin{cor}\label{1.4.4}
\quad $p$が素数なら$p\not|1$
\end{cor}
\begin{proof}
Cor\cref{1.4.3}より, 1の約数は$\pm1$のみなので, $p$が素数であれば, $p$は1の約数ではない, つまり1は$p$で割り切れない. 
\end{proof}

\begin{dfn}\label{1.4.5}
$a, b\in\mathbb{Z}$とする
\\(1)$a\ne0$または$b\ne0$であるとき, $a, b$の正の公約数の中で最大のものを最大公約数といい, $\gcd(a, b)$と書く, $\gcd(a, b)=1$ならば, $a, b$は互いに素であるという
\\(2)$a, b\ne0$なら, $a, b$の正の公倍数の中で最小のものを最小公倍数といい, $\lcm(a, b)$と書く
\end{dfn}

\cref{1.4.2}より約数の個数は有限個であり, 従って$a, b$の公約数の個数も有限個である. これにより有限個の公約数の中に最大のものがあるので, \cref{1.4.5}(1)は正当化される. (2)では, $|ab|$は$a, b$の公倍数であり, $|ab|$以下の正の整数の個数は有限なので, その中に$|ab|$の約数であって最小のものが存在する. よって\cref{1.4.5}(2)も正当化される. 

\begin{prop}\label{1.4.7}
(1)\quad $a|b, b|c$なら, $a|c$である
\\(2)\quad $m\ne0$なら, $a|b\iff am|bm$
\end{prop}

\begin{proof}
(1)\quad $a|b, b|c$より, $b=an, c=bm$となる$n, m\in\mathbb{Z}$が存在する. 代入して, $c=anm$よって$a|c$
\\(2)\quad まず$a|b \implies am|bm$を示す. 
\\$a|b \implies b=an(n\in\mathbb{Z})$ この両辺に$m\ne0$をかけて$bm=anm$, $n\in\mathbb{Z}$より$am|bm$
次に$am|bm \implies a|b$を示す. $am|bm \implies bm=amn(n\in\mathbb{Z})$この両辺を$m\ne0$で割ると$b=an$よって$a|b$
\\従って$m\ne0$なら, $a|b\iff am|bm$
\end{proof}

\vskip\baselineskip
数学的帰納法についての復習をしておこう. 
$P_n$を自然数$n$に対し与えられた数学的主張とするとき, 次のような論法を数学的帰納法, または単に帰納法という. 
\\数学的帰納法1 \quad $P_0$が正しく, $P_n$が正しいなら$P_n+1$が正しいとき, $P_n$は全ての$n$に対して正しい. 
\\また次のように使うこともある
\\数学的帰納法2 \quad $P_0$が正しく, 「全ての$m<n$に対し$P_m$が正しい」なら$P_m$が正しいとき, $P_n$は全ての$n$に対して正しい. 

\begin{prop}\label{1.4.8}
$n>1$が整数なら, $n$の約数で素数であるものがある
\end{prop}

\begin{proof}
$n$が素数でなければ, $1<m<n$を$n$の約数にもつ. ここで$P_n$を
\\$P_n$「$n>1$が整数なら, $n$の約数で素数であるものがある」とする. $P_2$は2が約数に素数2を持つので正しい. ここで$n>1$を整数とし, 「全ての$m<n$に対し, $P_m$が正しい」と仮定する. このとき$P_n$が正しいことを示す. 
\\仮定より全ての$m<n$に対し$P_m$が正しい, つまり$n$より小さい整数$m$は素数$p$を約数にもつ. 今. $m$は$n$の約数としてとってきたので, \textbf{prop1.4.7}(1)から$n$も$p$を約数に持つ. 
\end{proof}













